\section{Introduction}
\label{sec:introduction}

Many real-world problems may be abstracted as selecting a subset of the columns of a matrix representing stochastic observations or analytically exact data.
This paper focuses on a simple such problem that appears in interpretable learning and diversification.
Given a rank $D$ matrix $ X \in \mathbb R^{D \times P}$ with $P > D$, select a square submatrix $ X_{. S}$ where subset $ S \subset [P]$ satisfies $| S| = D$ that is as orthonormal as possible.

This problem arises in interpretable learning specifically because while the coordinate functions of a given feature space may have no intrinsic meaning, it is sometimes possible to generate a dictionary of interpretable features which may be considered as potential parametrizing coordinates.
When this is the case, selection of candidate interpretable features as coordinates can take the above form.
While implementations vary across data and algorithmic domains, identification of such coordinates generally aids mechanistic understanding, generative control, and statistical efficiency.

This paper shows that an adapted version of the algorithm in \citet{Koelle2024-no} leads to a convex procedure that can improve upon greedy approaches such as those in \citet{5895106, NEURIPS2019_6a10bbd4, Kohli2021-lr, Jones2007-uc} for finding isometries.
The insight leading to isometry pursuit is that multitask basis pursuit applied to an appropriately normalized $ X$ selects orthonormal submatrices.
Given vectors in $\mathbb R^D$, the normalization log-symmetrizes length and favors those closer to unit length, while basis pursuit favors those which are orthogonal.
Our results formalize this intuition within a limited setting, and show the usefulness of isometry pursuit as a trimming procedure prior to brute force search for diversification and interpretable coordinate selection.
We also introduce a novel ground truth objective function against which we measure the success of our algorithm, and discuss the reasonableness of the trimming procedure.

 \footnotetext[2]{Code is available at \url{https://github.com/sjkoelle/isometry-pursuit}.}